\documentclass{article}

\title{Pytorch Notes}

\begin{document}

\maketitle

\section{Pytorch}
\begin{description}
    \item[Tensor] Multidimensionales Array dessen Berechnungen auf GPU laufen können
    \item[Sample] Einzelnes Datenlement für Training (p.ex. ein Bild)
    \item[Dataset] Datenstruktur für Trainingsdaten \begin{itemize}
            \item Erlaubt Zugriff auf einzelne Samples
        \end{itemize}
    \item[Dataloader] Wrappt Dataset für effizienten Datenzugriff \begin{itemize}
            \item Erlaubt Zugriff auf Batches von Samples
        \end{itemize}
\end{description}

\section{ARIMA}
\begin{description}
    \item[AutoRegession] Werte werden anhand vorhergegangener vorhergesagt \begin{itemize}
            \item Zugehoeriger Parameter \textbf{p}, Anzahl der Werte die zurueck geblickt wird
        \end{itemize}
    \item[Integration] Nicht stationaere TS werden in stationaere umgewandelt \begin{itemize}
            \item Zugehoeriger Parameter \textbf{d}, Anzahl der Differenzierungen die noetig sind um die TS stationaer zu machen
        \end{itemize}
    \item[Moving Average] Anpassung der Vorhersagen durch Betrachtung vorhergegangener Fehler \begin{itemize}
            \item Zugehoeriger Parameter \textbf{q}, Anzahl der Fehler die zurueck geblickt wird
        \end{itemize}
    \item[ADF-Test] Testet ob TS stationaer
\end{description}

\end{document}
